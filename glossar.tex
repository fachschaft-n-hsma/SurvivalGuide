\section*{Glossar}
In diesem Kapitel versuchen wir wichtige Begriffe und Konzepte rund um die Hochschule und das Studieren generell zu erklären.

\subsection*{Im- und Exmatrikulation}\label{glos:exma}
Das Studium beginnt mit der Immatrikulation und endet mit der Exmatrikulation.
Nachdem deine Bewerbung bei der Hochschule Mannheim eingangen ist und angenommen wurde wirst du für deinen Studiengang immatrikuliert.
Du erhälst für dein gesamtes Studium an der Hochschule Mannheim eine Matrikelnummer.
Während der Dauer deines Studiums gilst du nach Überweisung des Semesterbeitrages als eingeschrieben.
%todo: abmildern
Du verlierst diesen Status entweder indem du Zwangsexmatrikuliert wirst oder dich freiwillig exmatrikulierst.
Eine Zwangsexmatrikulation erfolgt aus folgenden Gründe:
\begin{itemize}
    \item Nichtüberweisen der Semesterbeiträge
    \item Beim Durchfallen im Drittversuch bei einem Pflichtpfach
    \item Einer Überschreitung der Fachsemestergrenze im Grundstudium von vier Semestern
    \item Einer Überschreitung der Fachsemestergrenze im Hauptstudium von 10 Semestern
\end{itemize}
Am Ende deines Studiums musst du dich manuell exmatrikulieren.
\\
Wenn du nicht Immatrikuliert bist darfst du keine Vorlesungen oder Laborveranstaltungen besuchen.
Du verlierst außerdem den Prüfungsanspruch.
Das bedeutet du darfst keine Prüfungen an der Hochschule ablegen und dir werden auch sonst keine ECTS-Punkte gutgeschrieben.
\\

Bei einer vorzeitigen Exmatrikulation bekommst du eine Liste über deine abgelegten Prüfungen.
%todo: auslagern in "anrechnungen"
Mit dieser kannst du dir zum Beispiel an anderen Hochschulen Fächer anrechnen lassen.
Es gibt jedoch kein einheitliches System für eine Anrechung zwischen den Hochschulen, so dass eine Anrechnung meistens erst nach einer persönlichen Beurteilung eines Prüfers/einer Prüferin erfolgt.
Es empfiehlt sich also nicht zwischen den Hochschulen hin und her zu hüpfen.

\subsection*{Hochschulsemester und Fachsemester}
Hochschulsemester sind Zeiteinteilungen für dein Studium.
Ein Studienjahr teilt sich auf in ein Sommer- und ein Wintersemester.
Beide sind sechs Monate lang.
Ein Semester ist aufgeteilt in eine Vorlesungs- und eine Prüfungszeit.
In der Vorlesungszeit besuchst du Vorlesungen, Labore und wenn du willst Tutorien.
In der Prüfungszeit werden Klausuren geschrieben.
In dieser Zeit finden keine Vorlesungen statt, so dass du viel Zeit zum lernen hast.
Es empfiehlt sich ca. 6 Wochen vor Klausuren für diese zu lernen.
\\
Die Semester- und Klausurtermine stehen nicht von Anfang an fest.
Eine Übersicht findest du auf der Website der Hochschule Mannheim.
\footnote{https://www.hs-mannheim.de/studierende/termine-news-service/semestertermine.html}
\\\\
%todo: umformulieren
\textbf{Hochschulsemester} dienen wiederrum nur zur Orientierung wo du dich in deinem Studium befindest.

\subsection*{Rückmeldung}
Unter Rückmeldung ist die Überweisung des Semesterbeitrags gemeint.
Seit 2011 müssen Studierende in Baden-Württemberg mit Wohnsitz in der EU keine Studiengebühren mehr bezahlen.
Es bleibt jedoch der Semesterbeitrag um die Verwaltung zu finanzieren.
Informationen über die Höhe und die Bankdaten findest du auf der Website der des Service-Center-Studierende.
\footnote{https://www.hs-mannheim.de/studierende/service-center-studierende/gebuehren-beitraege.html}
Der Rückmeldezeitraum für das Sommersemester ist von Anfang Januar bis Ende Februar.
Für das Wintersemester von Anfang Juli bis Ende August.
Eine genaue Terminübersicht ist ebenfalls auf der oben verlinkten Website zu finden.
Erfolgt keine Rückmeldung wirst du rückwirkend Zwangsexmatrikuliert.
\\
Internationale Studierende deren Wohnsitz außerhalb der EU liegt müssen Studiengebühren bezahlen.
%todo: trennen: zweitstudium gilt für alle nicht nur für international
Im Erststudium betragen diese 1.500,00 € / Semester, im Zweitstudium 650,00 € / Semester.
Mehr Informationen auf der Website des Kultusministeriums \footnote{\url{https://mwk.baden-wuerttemberg.de/de/hochschulen-studium/studium/studienfinanzierung/gebuehren-fuer-internationale-studierende-und-zweitstudium/}}

\subsection*{POS - Prüfungsorganisationssystem}
Das POS \footnote{https://noten.hs-mannheim.de/} stellt neben Moodle die zentrale Verwaltungsschnittstelle für dein Studium dar.
Hier kannst du Semesterbescheinigungen herunterladen, dich an Prüfungen anmelden, abmelden oder deine Noten ansehen.
Außerdem kannst du deine Adresse und Telefonnummer ändern, falls du umgezogen bist oder den Mobilfunkanbieter gewechselt hast.
Semesterbescheinigungen brauchst du zum Beispiel für die Agentur für Arbeit, falls deine Eltern noch Kindergeld für dich bekommen oder für die Krankenkassen wenn du bei deinen Eltern Familien-Krankenversichert bist.

\subsection*{Härtefallantrag / Antrag auf Studienzeitverlängerung}
Es ist also passiert. 
Du hast die Fachsemestergrenzen verletzt? Dir droht die Zwangsexmatrikulation.
Wenn du einen triftigen Grund wie z.B. außere Umstände oder  eine attestierte Krankheit hattest kannst du einen Härtefallantrag bzw. einen Antrag auf Studienzeitverlängerung stellen.
%todo: Studiengangsleitung vorher konsultieren. Dokument nicht alleine ausfüllen.
Diesen musst du an das Prüfungsamt senden, welches dann über deinen Antrag entscheidet und ihm ggf. stattgibt.
Eine Liste aller Anträge gibt es beim Service-Center-Studierende.
Härtefallanträge gibt es nicht für Drittversuche, wie das bei anderen Hochschulen oft üblich ist.
Nur für die maximale Studiendauer.
Bei einem nichbestandenen Drittversuch erfolgt zwanagsläufig die Exmatrikulation.
\footnote{https://www.hs-mannheim.de/studierende/service-center-studierende/dokumente-download.html}
Wir empfehlen dir bei der feststellung einer chronischen Krankheit ein Urlaubssemester zu beantragen. 
Den Vorgang haben unter ``\hyperref[pruef:urlaub]{\nameref{pruef:urlaub}}'' schonmal geklärt.

\subsection*{ECTS}
Mit ECTS sind meistens ECTS-\textit{Punkte} gemeint.
ECTS-Punkte werden während des Studiums gesammelt.
% 210 nur an der HS. Eher rausnehmen.
Für den Bachelor-Abschluss werden hierfür 210, für den Master 90 ECTS Punkte gesammelt werden.
Prüfungen sind dabei unterschiedliche viele Punkte Wert.
Manche Prüfungen die z.B. aus zwei Vorlesungen bestehen geben auch schonmal 8 ECTS, während die Meisten sonst eher 5 geben.
%todo: verweis gewichtung für die endnote aufs modulhandbuch
Über die Anzahl der ECTS pro Fach können daher manchmal auch Rückschlüsse auf den Umfang von Fächern abgeleitet werden.

\subsection*{Studien- und Prüfungsordnung}
Die Studien- und Prüfungsordnung (kurz: StuPo) beschreibt alle rechtlichen wichtigen Punkte zu deinem Studium.
Darin ist zum Beispiel die Gliederung, Klausuren, Wiederholungsbedingungen und Anrechnungen beschrieben.
Es gibt eine StuPo für Bachelor- und eine für Masterstudierende.
\footnote{https://www.hs-mannheim.de/studierende/service-center-studierende/studien-und-pruefungsordnungen.html}

\subsection*{Bachelorvorprüfung}
Die Bachelorvorprüfung ist in diesem Sinne keine Prüfung.
Sie gilt ab dem Zeitpunkt als Bestanden ab dem alle Grundstudiumsfächer bestanden wurden.
Ab dann befindet man sich offiziell im Hauptstudium.
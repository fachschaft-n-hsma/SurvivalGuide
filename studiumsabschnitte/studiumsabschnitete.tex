\chapter{Unterschiede Haupt- und Grundstudium}


%@Marius: Wie hast du "Fachsemester" und "Semester" definiert?
%


\section{Grundstudium}

\textbf{Dauer:}\\
Das Grundstudium sind die \textbf{Fachsemester} (die Zahl, welche hinter eurer Mathe/Physik/E-Technik Vorlesung steht) eins und zwei.\\
Für diese \textbf{Fachsemester} habt ihr vier \textbf{Hochschulsemester} (die Semester, welche ihr insgesamt an der Hochschule seid) Zeit.\\
In diesen vier \textbf{Hochschulsemester} MÜSST ihr alle Prüfungen der ersten beiden Fachsemester bestanden haben.\\
Bei Überschreitung der Zeit solltet ihr die Studienberatung aufsuchen.\\
%Querverweis auf Kontakte Studiumsberatung

\textbf{Anmeldung Prüfungen:}\\
Im Grundstudium werdet ihr bei den Prüfungen automatisch angemeldet.\\
Es ist aber empfehlenswert im POS nachzuschauen, ob die automatische Anmeldung funktioniert hat.\\
Abmeldungen sind im Grundstudium nur über den Studienberater möglich.\\
Speichert euch dann am besten die Prüfungsanmeldung als PDF und bringt diese mit zur Klausur.\\
\textbf{Achtet auf die Fristen der Prüfungsabmeldung!}\\ %Querverweis auf ->Fristen
Diese endet im Regelfall vier Wochen vor der Klausurenphase.\\
Schaut vor der Prüfungsphase in euren Maileingang, dort bekommt ihr alle wichtigen Informationen.\\
Es ist möglich Prüfungen aus dem Hauptstudium zu schreiben, vorrausgesetzt Ihr habt nicht mehr als drei Prüfungen aus dem Grundstudium offen.\\

\textbf{Wertung:}\\
Die Prüfungsleistungen werden nicht in der Bachelornote gewichtet, müssen aber bestanden werden.\\ 

\section{Hauptstudium}
%todo: 10 Semsester = dauer(grund + haupt) 
\textbf{Dauer:}\\
Das Hauptstudium beginnt \textbf{im dritten Fachsemester}.\\
Dafür habt ihr insgesamt \textbf{10 Hochschulsemester Zeit}.\\
Auch hier gilt, in Absprache mit der Studienberatung kann die Regelstudienzeit verlängert werden.\\

\textbf{An- und Abmeldung Prüfungen:}\\
Im Hauptstudium müsst ihr euch für alle Prüfungen im POS selber anmelden.\\
Die Ausnahmen sind Wiederholungsprüfungen.\\
Hierzu seid ihr automatisch angemeldet, eine Abmeldung ist aber grundsätzlich möglich.\\ %Wirklich?
Eine Prüfungsabmeldung ist bis zu 24 Stunden vor der Klausur möglich.\\

\textbf{Wertung:}\\
Die Prüfungsleistungen werden in der Bachelornote gewichtet, die Gewichtung könnt ihr in der \glqq Studien- und Prüfungsordnung \grqq nachschauen.\\

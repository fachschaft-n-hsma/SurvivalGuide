\chapter{Prüfungen}

Vorlesungen werden in der Regel am Ende der Vorlesungszeit durch eine Prüfung abgeschlossen (meist schriftliche Klausur).\\
Üblicherweise werden die Form, Punkteverteilung, Teilnahmebedingungen (zum Beispiel Pflichtlabore) und mehr in den ersten
Vorlesungen des Semesters besprochen.\\

Wir geben euch hier einen kurzen Überblick, detailliert  findet ihr alle Infos im Moodlekurs:
\textbf{Informationen zum Studium \(\rightarrow\) Fakultätsinfos \(\rightarrow\) Merkblatt Bachelor}.\\


\section{Wiederholung}
%Falls Ihr eine Prüfung nicht bestehen solltet, müsst Ihr vor dem S Gebäude auf dem Parrkplatz nackt einen Tricher im Handstand inhalieren.\\ %Idee?
Falls Ihr eine Prüfung nicht bestehen solltet, steckt den Kopf nicht in den Sand, Ihr werdet automatisch im nächsten Semester zur Prüfung angemeldet.\\
Im Anschluss habt Ihr die Möglichkeit einen zweiten Versuch zu wagen.\\
Sollte auch dieser nicht erfolgreich sein, ist ein Gespräch mit dem/der Dozenten:in für eine zweite Wiederholung verpflichtend, bei mehreren Fehlversuchen
in einem Semester mit der Studienberatung.\\

\section{Beurlaubung}
Aus wichtigem Grund (Krankheiten, familiäre Gründe etc) könnt Ihr beim Prüfungsamt ein Urlaubssemester beantragen.\\
In einem Urlaubssemester dürft Ihr keine Prüfungen schreiben, es zählt auch nicht in die Regelstudienzeit mit rein.\\ 
Ein Urlaubssemester muss frühzeitig beantragt werden.\\
Sich in einem Semester von allen Prüfungen abzumelden, gilt also nicht als Urlaubssemester.\\

\section{Annerkennung geschriebener Prüfungen und Ausbildungen}
Für gewöhnlich bekommt Ihr am Anfang jedes Semesters eine Mail zum Thema
\textbf{Anerkennung von Prüfungsleistungen im Grundstudium aus vorherigem Studium oder Ausbildung}.\\
Beachtet, dass der Antrag in den ersten zwei Wochen des Semesters gestellt werden muss.\\


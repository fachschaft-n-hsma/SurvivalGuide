\section*{Wichtige Ansprechpartner}
Die Wenigsten kommen ganz ohne Probleme durchs Studium.
An unserer Fakultät gibt es daher für jedes Problem unterschiedliche Ansprechpartner.
\\
Diese wechseln jedoch oft durch, so dass wir hier nicht alle herunter schreiben können.
Eine vollständige Liste findest du auf der Website der Fakultät \footnote{https://www.inftech.hs-mannheim.de/fakultaet/ansprechpartner/funktionen.html}

\subsection*{Studiengangsleitung}
Für Fragen während des Grundstudiums kontaktiere bitte die Studiumsberaterin / den Studienberater für's Grundstudium.
Diese hilft dir zum Beispiel weiter, wenn du den Studiengang wechseln willst. 
Bei einer Prüfungsabmeldung von Grundstudiumsprüfungen \textit{muss} man sogar eine Beratung in Anspruch nehmen.
Eine Abmeldung über das POS reicht nicht aus.
Dies gilt auch, falls es zu mehreren Fehlversuchen im Grundstudium kommt.
\\
Hat man es erfolgreich ins Hauptstudium geschafft gibt es pro Studiengang eine Studiendekanin bzw. einen Studiendekan, die/der ab dann für alle Anfragen zuständig ist.
\\
Für Masterstdierende gibt es nochmal extra einen/eine Studiendekan/in.

\subsection*{Auslandsbeauftrate/r}
Die Hochschule bietet zahlreiche Austauschuniversitäten in ganz Europa und darüber hinaus an.
Acht dieser europäischen Universitäten nehmen am Erasmus+-Programm teil.
In diesem Fall bekommen Studierende bis zu 450 € monatliche Förderung im Ausland damit man sich zum Beispiel eine Unterkunft finanzieren kann. 
\\
Ein Auslandssemester muss nicht immer ein Theorie-Semester sein.
Auch Praxissemester, Studienarbeiten oder Abschlussarbeiten sind möglich.
\\
Für eine umfangreiche Beratung empfiehlt sich ein Termin mit dem/der Auslandsbeauftragte/n oder ein Blick auf die Website des International Office. \footnote{https://www.hs-mannheim.de/die-hochschule/internationales/international-office/}

\subsection*{Sekreteriat}
Weißt du nicht wer zuständig ist für dein Anliegen?
Dann schreib einfache eine email direkt an das Sekreteriat.
